\documentclass[a4paper,12pt]{article} %page properties
\usepackage[utf8]{inputenc}
\usepackage[russian]{babel}
\usepackage{fullpage}
\usepackage{hyperref} %for urls
\usepackage{tabularx}
\usepackage{longtable} %for tables on multiple pages

\begin{document}
% % % % % % % % % % % % % % % % % % % % % % % % % % % % % % % % %
\tableofcontents

% % % % % % % % % % % % % % % % % % % % % % % % % % % % % % % % %
\newpage
\section{Main List}
\begin{longtable}{|l|p{6cm}|c|c|l|}
	\hline
	code & f.title & b.date & e.date & Note \\
	\hline
	\multicolumn{5}{|l|}{edX}\\
	\hline
	IT.1.1x & Introduction to Programming with Java, part 1 &  & 2016-07-01 & Self-paced\\
	\hline
	&  &  &  &  \\
	\hline
	PH525.1x & Data Analysis for Life Sciences 1: Statistics and R & 2015-10-15 & 2016-09-15 & Self-paced \\
	\hline
	PH525.2x & Data Analysis for Life Sciences 2: Introduction to Linear Models and Matrix Algebra & 2015-11-15 & 2016-09-15 & Self-paced \\
	\hline
	PH525.3x & Data Analysis for Life Sciences 3: Statistical Inference and Modeling for High-throughput Experiments & 2015-12-15 &  & Self-paced \\
	\hline
	PH525.4x & Data Analysis for Life Sciences 4: High-Dimensional Data Analysis & 2016-01-15 &  & Self-paced \\
	\hline
	PH525.5x & Data Analysis for Life Sciences 5: Introduction to Bioconductor: Annotation and Analysis of Genomes and Genomic Assays & 2016-02-15 &  & Self-paced \\
	\hline
	PH525.6x & Data Analysis for Life Sciences 6: High-performance Computing for Reproducible Genomics & 2016-03-15 &  & Self-paced \\
	\hline
	PH525.7x & Data Analysis for Life Sciences 7: Case Studies in Functional Genomics & 2016-04-15 &  & Self-paced \\
	\hline
	&  &  &  &  \\
	\hline
	LFS101x.2 & Introduction to Linux &  &  & Self-paced \\
	\hline
	&  &  &  &  \\
	\hline
	\multicolumn{5}{|l|}{Coursera}\\
	\hline
	&  &  &  &  \\
	\hline
	&  &  &  &  \\
	\hline
	&  &  &  &  \\
	\hline
	\multicolumn{5}{|l|}{Stanford}\\
	\hline
	& Statistical learning & 2016-01-12 & 2016-04-04 &  \\
	\hline
	&  &  &  &  \\
	\hline
	\multicolumn{5}{|l|}{Propellers}\\
	\hline
	& 3D-мультфильм с нуля &  &  &  \\
	\hline
	& Blender Level-Up &  &  &  \\
	\hline
	&  &  &  &  \\
	\hline
\end{longtable}

% % % % % % % % % % % % % % % % % % % % % % % % % % % % % % % % %
\newpage
\section{Graphic}
\subsection{Propellers}
\subsubsection{3d}
\begin{longtable}{|l|p{11cm}|l|l|}
	\hline
	\# & Topic & Len & Note \\
	\hline
	1 &  &  &  \\
	a & Интерфейс &  &  \\
	b & Редактирование &  &  \\
	c & Видеомонтаж &  &  \\
	\hline
	2 &  &  &  \\
	a & Архитектура &  &  \\
	b & Материалы &  &  \\
	c & Моделирование &  &  \\
	\hline
	3 & Модификаторы &  &  \\
	\hline
	4 &  &  &  \\
	a & Оснастка, часть1 &  &  \\
	b & Оснастка, часть2 &  &  \\
	\hline
	5 &  &  &  \\
	a & Скелет &  &  \\
	b & Модификатор Skin &  &  \\
	\hline
	6 &  &  &  \\
	a & Ключи формы &  &  \\
	b & Гуманоидный риг &  &  \\
	\hline
	7 &  &  &  \\
	a & Шейдеры Internal &  &  \\
	b & Шейдеры Cycles &  &  \\
	\hline
	8 &  &  &  \\
	a & UV развертка &  &  \\
	b & Рисование текстур &  &  \\
	c & Рендер UV &  &  \\
	\hline
	9 &  &  &  \\
	a & Кривые анимации &  &  \\
	b & Работа с ключами &  &  \\
	c & Скелетная анимация &  &  \\
	\hline
	10 & 12 правил анимации &  &  \\
	\hline
	11 &  &  &  \\
	a & Основы линкования &  &  \\
	b & Типы адресов &  &  \\
	c & Сложное линкование &  &  \\
	d & Связи датаблоков &  &  \\
	\hline
	12 & Композитинг &  &  \\
	\hline
	13 &  &  &  \\
	a & Техника безопасности &  &  \\
	b & Жизнь после курса &  &  \\
	\hline
	&  &  &  \\
	\hline
\end{longtable}

\subsubsection{Level-Up}
\begin{longtable}{|l|p{11cm}|l|l|}
	\hline
	\# & Topic & Len & Note \\
	\hline
	1 &  &  &  \\
	a & Хоткеи &  &  \\
	b & Скрытые функции &  &  \\
	\hline
	2 &  &  &  \\
	a & Азбука NLA &  &  \\
	b & Применение NLA &  &  \\
	\hline
	3 &  &  &  \\
	a & Анимация мяча &  &  \\
	\hline
	&  &  &  \\
	\hline
\end{longtable}


% % % % % % % % % % % % % % % % % % % % % % % % % % % % % % % % %
\newpage
\section{Data Analysis}

\subsection{Statistical Learning}

\begin{longtable}{|l|p{11cm}|l|c|l|}
	\hline
	\# & Topic & Len & Ass & Date \\
	\hline
	1 & \multicolumn{4}{|l|}{12-01-2016 \hfill Introduction and  \hfill 04-04-2016}\\
	\hline
	1.1 & Opening remarks & 18-19 & --- & 17-01-2016 \\
	\hline
	1.2 & Examples and Framework & 12-13 & 2/2 & 17-01-2016 \\
	\hline
	& & & &  \\
	\hline
	2 & \multicolumn{4}{|l|}{12-01-2016 \hfill Overview of Statistical Learning \hfill 04-04-2016}\\
	\hline
	2.1 & Introduction to Regression Models & 11-42 & 1/1 & 17-01-2016 \\
	\hline
	2.2 & Dimensionality and Structured Models & 11-41 & 1/1 & 17-01-2016 \\
	\hline
	2.3 & Model Selection and Bias-Variance Tradeoff & 10-05 & 2/2 & 17-01-2016 \\
	\hline
	2.4 & Classification & 15-38 & 1/1 & 17-01-2016 \\
	\hline
	2.R & Introduction to R & 14-13 & 1/1 & 17-01-2016 \\
	\hline
	& ch quiz & & 4/4 & 17-01-2016 \\
	\hline
	& & & &  \\
	\hline
	3 & \multicolumn{4}{|l|}{16-01-2016 \hfill Linear Regression \hfill 04-04-2016}\\
	\hline
	3.1 & & & &  \\
	\hline
	3.2 & & & &  \\
	\hline
	3.3 & & & &  \\
	\hline
	3.4 & & & &  \\
	\hline
	3.5 & & & &  \\
	\hline
	3.R & & & &  \\
	\hline
	& & & &  \\
	\hline
	4 & \multicolumn{4}{|l|}{23-01-2016 \hfill Classification \hfill 04-04-2016}\\
	\hline
	& & & &  \\
	\hline
	5 & \multicolumn{4}{|l|}{30-01-2016 \hfill Resampling Methods \hfill 04-04-2016}\\
	\hline
	& & & &  \\
	\hline
	6 & \multicolumn{4}{|l|}{06-02-2016 \hfill Linear Model Selection and Regularization \hfill 04-04-2016}\\
	\hline
	& & & &  \\
	\hline
	7 & \multicolumn{4}{|l|}{13-02-2016 \hfill Moving Beyond Linearity \hfill 04-04-2016}\\
	\hline
	& & & &  \\
	\hline
	8 & \multicolumn{4}{|l|}{20-02-2016 \hfill Tree-based Methods \hfill 04-04-2016}\\
	\hline
	& & & &  \\
	\hline
	9 & \multicolumn{4}{|l|}{27-02-2016 \hfill Support Vector Machines \hfill 04-04-2016}\\
	\hline
	& & & &  \\
	\hline
	10 & \multicolumn{4}{|l|}{05-03-2016 \hfill Unsupervised Learning \hfill 04-04-2016}\\
	\hline
	& & & &  \\
	\hline
\end{longtable}



% % % % % % % % % % % % % % % % % % % % % % % % % % % % % % % % %
\newpage
\section{Languages}
\subsection{IT.1.1x Introduction to Programming with Java, part 1}

\begin{longtable}{|l|p{11cm}|l|l|l|}
	\hline
	\# & Topic & Len & Ass & Date \\
	\hline
	0 & \multicolumn{4}{|l|}{Introduction}\\
	\hline
	& & & &  \\
	\hline
	1 & \multicolumn{4}{|l|}{From the Calculator to the Computer}\\
	\hline
	& & & &  \\
	\hline
	2 & \multicolumn{4}{|l|}{State Transformation}\\
	\hline
	& & & &  \\
	\hline
	3 & \multicolumn{4}{|l|}{Functional Abstraction}\\
	\hline
	& & & &  \\
	\hline
	4 & \multicolumn{4}{|l|}{Object Encapsulation}\\
	\hline
	& & & &  \\
	\hline
	5 & \multicolumn{4}{|l|}{Packaging}\\
	\hline
	& & & &  \\
	\hline
\end{longtable}

% % % % % % % % % % % % % % % % % % % % % % % % % % % % % % % % %
\end{document}